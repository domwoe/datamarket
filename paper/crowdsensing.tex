% This is "sig-alternate.tex" V2.0 May 2012
% This file should be compiled with V2.5 of "sig-alternate.cls" May 2012
%
% This example file demonstrates the use of the 'sig-alternate.cls'
% V2.5 LaTeX2e document class file. It is for those submitting
% articles to ACM Conference Proceedings WHO DO NOT WISH TO
% STRICTLY ADHERE TO THE SIGS (PUBS-BOARD-ENDORSED) STYLE.
% The 'sig-alternate.cls' file will produce a similar-looking,
% albeit, 'tighter' paper resulting in, invariably, fewer pages.
%
% ----------------------------------------------------------------------------------------------------------------
% This .tex file (and associated .cls V2.5) produces:
%       1) The Permission Statement
%       2) The Conference (location) Info information
%       3) The Copyright Line with ACM data
%       4) NO page numbers
%
% as against the acm_proc_article-sp.cls file which
% DOES NOT produce 1) thru' 3) above.
%
% Using 'sig-alternate.cls' you have control, however, from within
% the source .tex file, over both the CopyrightYear
% (defaulted to 200X) and the ACM Copyright Data
% (defaulted to X-XXXXX-XX-X/XX/XX).
% e.g.
% \CopyrightYear{2007} will cause 2007 to appear in the copyright line.
% \crdata{0-12345-67-8/90/12} will cause 0-12345-67-8/90/12 to appear in the copyright line.
%
% ---------------------------------------------------------------------------------------------------------------
% This .tex source is an example which *does* use
% the .bib file (from which the .bbl file % is produced).
% REMEMBER HOWEVER: After having produced the .bbl file,
% and prior to final submission, you *NEED* to 'insert'
% your .bbl file into your source .tex file so as to provide
% ONE 'self-contained' source file.
%
% ================= IF YOU HAVE QUESTIONS =======================
% Questions regarding the SIGS styles, SIGS policies and
% procedures, Conferences etc. should be sent to
% Adrienne Griscti (griscti@acm.org)
%
% Technical questions _only_ to
% Gerald Murray (murray@hq.acm.org)
% ===============================================================
%
% For tracking purposes - this is V2.0 - May 2012

\documentclass{sig-alternate}

\begin{document}
%
% --- Author Metadata here ---
\conferenceinfo{WOODSTOCK}{'97 El Paso, Texas USA}
%\CopyrightYear{2007} % Allows default copyright year (20XX) to be over-ridden - IF NEED BE.
%\crdata{0-12345-67-8/90/01}  % Allows default copyright data (0-89791-88-6/97/05) to be over-ridden - IF NEED BE.
% --- End of Author Metadata ---

\title{Incentivizing mobile crowdsensing using Bitcoin micropayments}


%
% You need the command \numberofauthors to handle the 'placement
% and alignment' of the authors beneath the title.
%
% For aesthetic reasons, we recommend 'three authors at a time'
% i.e. three 'name/affiliation blocks' be placed beneath the title.
%
% NOTE: You are NOT restricted in how many 'rows' of
% "name/affiliations" may appear. We just ask that you restrict
% the number of 'columns' to three.
%
% Because of the available 'opening page real-estate'
% we ask you to refrain from putting more than six authors
% (two rows with three columns) beneath the article title.
% More than six makes the first-page appear very cluttered indeed.
%
% Use the \alignauthor commands to handle the names
% and affiliations for an 'aesthetic maximum' of six authors.
% Add names, affiliations, addresses for
% the seventh etc. author(s) as the argument for the
% \additionalauthors command.
% These 'additional authors' will be output/set for you
% without further effort on your part as the last section in
% the body of your article BEFORE References or any Appendices.

\numberofauthors{1} %  in this sample file, there are a *total*
% of EIGHT authors. SIX appear on the 'first-page' (for formatting
% reasons) and the remaining two appear in the \additionalauthors section.
%
\author{
% You can go ahead and credit any number of authors here,
% e.g. one 'row of three' or two rows (consisting of one row of three
% and a second row of one, two or three).
%
% The command \alignauthor (no curly braces needed) should
% precede each author name, affiliation/snail-mail address and
% e-mail address. Additionally, tag each line of
% affiliation/address with \affaddr, and tag the
% e-mail address with \email.
%
% 1st. author
% \alignauthor
% Dominic W\"orner\\
%        \affaddr{Department of Management, Technology and Economics}\\
%        \affaddr{ETH Zurich}\\
%        \affaddr{Switzerland}\\
%        \email{dwoerner@ethz.ch}
 }
% There's nothing stopping you putting the seventh, eighth, etc.
% author on the opening page (as the 'third row') but we ask,
% for aesthetic reasons that you place these 'additional authors'
% in the \additional authors block, viz.
% Just remember to make sure that the TOTAL number of authors
% is the number that will appear on the first page PLUS the
% number that will appear in the \additionalauthors section.

\maketitle
\begin{abstract}
Contribution
\begin{itemize}
\item Incentive mechanism for participating in mobile crowdsensing based on Bitcoin micropayments
\item Discussion of the advantages of using Bitcoin in comparison to previous monetary incentive mechanisms such as Mturk (20\% fees, complex process to become a worker) or credits (trust, costs, No personal sign up anywhere,..)
\item Presenting an Implementation
\item Evaluation and discussion of performance, security, privacy and cost (?) aspects
\item Presenting opportunities and challenges for future work
\end{itemize}

\end{abstract}

% A category with the (minimum) three required fields
% \category{H.4}{Information Systems Applications}{Miscellaneous}
% %A category including the fourth, optional field follows...
% \category{D.2.8}{Software Engineering}{Metrics}[complexity measures, performance measures]

% \terms{Theory}

% \keywords{ACM proceedings, \LaTeX, text tagging}

\section{Introduction}
Crowdsensing - Participatory sensing - Opportunistic sensing - Need for incentives - Monetary Incentives - Payment infrastructure (MTURK / Pay Pal) - Bitcoin

\section{Related work}

\subsection{Mobile crowdsensing in general}

In the vast galaxy of crowdsourcing activities, crowd-sensing consists in
using users’ cellphones for collecting large sets of data. In this chapter, we present the
APISENSE distributed crowd-sensing platform. In particular, APISENSE provides a
participative environment to easily deploy sensing experiments in the wild. Beyond
the scientific contributions of this platform, the technical originality of APISENSE
lies in its Cloud orientation, which is built on top of the soCloud distributed multicloud
platform, and the remote deployment of scripts within the mobile devices of
the participants. We validate this solution by reporting on various crowd-sensing
experiments we deployed using Android smartphones and comparing our solution
to existing crowd-sensing platforms. \cite{Haderer:2015bx}

With the surging of smartphone sensing, wireless networking, and mobile social networking techniques, Mobile Crowd Sensing and Computing (MCSC) has become a promising paradigm for cross-space and large-scale sensing. MCSC extends the vision of participatory sensing by leveraging both participatory sensory data from mobile devices (offline) and user-contributed data from mobile social networking services (online). Further, it explores the complementary roles and presents the fusion/collaboration of machine and human intelligence in the crowd sensing and computing processes. This article characterizes the unique features and novel application areas of MCSC and proposes a reference framework for building human-in-the-loop MCSC systems. We further clarify the complementary nature of human and machine intelligence and envision the potential of deep-fused human--machine systems. We conclude by discussing the limitations, open issues, and research opportunities of MCSC. \cite{guo2015mobile}

The ubiquity of smartphones and their on-board sensing capabilities motivates crowd-sensing, a capability which harnesses the power of crowds to collect sensor data from a large number of mobile phone users. Unlike previous work on wireless sensing, crowd-sensing poses several novel requirements: support for humans-in-the-loop to trigger sensing actions or review results, the need for incentives, as well as privacy and security. In this paper, we design and implement Medusa, a novel programming framework for crowd sensing that satisfies these requirements. Medusa provides high-level abstractions for specifying the steps required to complete a crowd-sensing task, and employs a distributed runtime system that coordinates the execution of these tasks between smartphones and a cluster on the cloud. We have implemented ten crowd-sensing tasks on a prototype of Medusa. We find that Medusa task descriptions are two orders of magnitude smaller than standalone systems required to implement those crowd-sensing tasks, and the runtime has low overhead and is robust to dynamics and resource attacks. \cite{Ra12a} -- uses Mturk for incentives/payments

\subsection{Incentive mechanisms}

Mobile sensing exploits data contributed by mobile users (e.g., via their smart phones) to make sophisticated infer- ences about people and their surrounding and thus can be applied to environmental monitoring, traffic monitoring and healthcare. However, the large-scale deployment of mobile sensing applica- tions is hindered by the lack of incentives for users to participate and the concerns on possible privacy leakage. Although incentive and privacy have been addressed separately in mobile sensing, it is still an open problem to address them simultaneously. In this paper, we propose two privacy-aware incentive schemes for mobile sensing to promote user participation. These schemes allow each mobile user to earn credits by contributing data without leaking which data it has contributed, and at the same time ensure that dishonest users cannot abuse the system to earn unlimited amount of credits. The first scheme considers scenarios where a trusted third party (TTP) is available. It relies on the TTP to protect user privacy, and thus has very low computation and storage cost at each mobile user. The second scheme removes the assumption of TTP and applies blind signature and commitment techniques to protect user privacy. \cite{li2013providing}

With the prosperity of smart devices, crowdsourcing has emerged as a new computing/networking paradigm. Through the crowdsourcing platform, service requesters can buy service from service providers. An important component of crowdsourcing is its incentive mechanism. We study three models of crowdsourcing, which involve cooperation and competition among the service providers. Our simplest model generalizes the well-known user-centric model studied in a recent Mobicom paper. We design an incentive mechanism for each of the three models, and prove that these incentive mechanisms are individually rational, budget-balanced, computationally efficient, and truthful. \cite{7218676}

Recent years have witnessed the fast proliferation of mobile devices (e.g., smartphones and wearable devices) in people’s lives. In addition, these devices possess powerful computation and communication capabilities, and are equipped with various built-in functional sensors. The large quantity and advanced functionalities of mobile devices have created a new interface between human beings and environments. Many mobile crowd sensing applications have thus been designed, which recruit normal users to contribute their resources for sensing tasks. To guarantee good performance of such applications, it’s essential to recruit sufficient participants. Thus, how to effectively and efficiently motivate normal users draws growing attention in the research community. This paper surveys diverse strategies that are proposed in the literature to provide incentives for stimulating users to participate in mobile crowd sensing applications. The incentives are divided into three categories: entertainment, service, and money. Entertainment means that sensing tasks are turned into playable games in order to attract participants. Incentives of service exchanging are inspired by the principle of mutual benefits. Monetary incentives give participants payments for their contributions. We describe literature works of each type comprehensively and summarize them in a compact form. Further challenges and promising future directions concerning incentive mechanism design are also discussed. \cite{7065282}

\subsection{Privacy}
Mobile participatory sensing has opened the doors to numerous sensing scenarios that were unimaginable few years ago. In absence of protection mechanisms, most of these applications may however endanger the privacy of the participants and end users. In this manuscript, we highlight both sources and targets of these threats to privacy and analyze how they are addressed in recent privacy-preserving mechanisms tailored to the characteristics of participatory sensing. We further provide an overview of current trends and future research challenges in this area. \cite{Christin2015}

\subsection{Reputation}

Participatory sensing is an emerging paradigm in which citizens everywhere voluntarily use their computational devices to capture and share sensed data from their surrounding environments in order to monitor and analyze some phenomenon (e.g., weather, road traffic, pollution, etc.). Interest in participatory sensing systems has risen since a large mobile sensor network can now be opportunistically constructed with much less cost and effort than it was the case a decade ago. However, relying on citizens who share their contributions raises many challenges. Participants can disrupt the system by contributing corrupted, fabricated, or erroneous data. Consequently, monitoring the participants’ behavior in order to estimate their honesty is an essential requirement. This enables to evaluate the veracity and accuracy of participants’ contributions and therefore, to build robust and reliable participatory sensing systems. Recently, several trust and reputation systems have been proposed to trace participants’ behavior in these systems. This survey presents a study and analysis of existing trust systems in participatory sensing applications. First, we study the nature of participatory sensing applications by surveying existing systems and outlining their common features. We then analyze the main vulnerabilities and attacks that can be launched in these systems. Furthermore, we discuss the concept of trust and we introduce a classification of existing trust systems. The two main classes of trust assessment methods for participatory sensing (i.e. Trusted Platform Module and reputation) are discussed. In addition, we analyze the merits as well as the limitations of each of them. We then derive a comparative study of several existing trust systems for participatory sensing. From this study, we identify many trust problems that have not been solved and many attacks have not been addressed yet in the literature. Finally, we list future research directions regarding trust management in participatory sensing systems. \cite{Mousa201549}

\section{Bitcoin}

	\subsection{Technical introduction}

	\begin{itemize}
	\item Peer-to-peer network
	\item Cryptographic signing of transactions
	\item Blockchain, a distributed ledger based consensus by proof of work
	\item Structure of transactions (challenge and claiming scripts)
	\item But: Transaction costs, consensus process -> payments aren't instant
	\end{itemize}

	\subsection{Important characteristics}
	\begin{itemize}
	\item Ad-hoc generation of accounts by anyone and anything
	\item Pseudonymous
	\item Global: frictionless payments across boarders
	\item Irreversibility of payments
	\item But: Transaction costs, consensus process -> payments aren't instant
	\end{itemize}

	\subsection{Instant Bitcoin micropayments}

    	\subsubsection{One-to-one payment channels}

		\subsubsection{N-to-N mediated payment channels}


\section{System Design}

	\subsection{Requirements or Features ?}
	\begin{itemize}
	\item Low barrier for participation
	\item Focus on data provider / data provider centric
	\item Limit power of platform provider
	\end{itemize}

	\subsection{Architecture}

\section{Implementation}

	\subsection{HTLC payment client}

	\subsection{HTLC payment server}

	\subsection{Mobile sensing application}

	\begin{itemize}
	\item Physical sensors
		\begin{itemize}
		\item Motion
		\item Acceleration
		\item Magnetic Field
		\item Network strength
		\item Barometric pressure
		\item Temperature
		\item Relative humidity
		\item Light
		\item Cellular network strength
		\item Location
		\end{itemize}
	\item Virtual sensors
		\begin{itemize}
		\item Wifi networks
		\item Bluetooth Beacons
		\item Data rate
		\item Transportation mode
		\item Installed apps
		\item Running apps
		\item Number of contacts
		\end{itemize}
	\end{itemize}


	\subsection{Querying and buying application}

	\subsection{Hub}

\section{Evaluation}

	\subsection{Performance}
	\begin{itemize}
	\item Performance of each task (query, buy, initialize payment channel..)
	\item Would be interesting to disintegrate cryptographic operations and network (round trip) times
	\item Contrast on-chain payment with off-chain payment
	\end{itemize}


	\subsection{Security}
	\begin{itemize}
	\item Security analysis of HTLC
	\end{itemize}

	\subsection{Privacy}
		\subsubsection{Data providers}
		\begin{itemize}
		\item IP address / device id
		\item location
		\item Inference based on bought data
		\end{itemize}
		\subsubsection{Buyers}
		\begin{itemize}
		\item IP address / device id
		\end{itemize}


\section{Discussion}

	\subsection{Privacy}
	\begin{itemize}
	\item{TOR}
	\end{itemize}

	\subsection{Quality of data and fraud}
	\begin{itemize}
	\item Fake phone
	\item Fake data
	\end{itemize}

	\subsection{Locking of funds / Capital investment}
	\begin{itemize}
	\item Hub operator needs to lock funds with every sensor
	\end{itemize}

	\subsection{Bitcoin transaction malleability}

\section{Future Work}

	\subsection{Internet of Things}

	\begin{itemize}
	\item Implementation on constrained devices / sensors
	\item Ideal for embedded/head-less devices
	\end{itemize}

	\subsection{Real-time data markets}

	\subsection{Decentralized Bitcoin payment channel networks}

	\begin{itemize}
	\item Duplex Payment Channels
	\item Lightning Network
	\end{itemize}

	\subsection{Reputation}

	\subsection{Preserving privacy}

	\begin{itemize}
	\item Enigma
	\end{itemize}

	\subsection{From opportunistic sensing to participatory sensing}

\section{Conclusions}

%ACKNOWLEDGMENTS are optional
\section{Acknowledgments}

%
% The following two commands are all you need in the
% initial runs of your .tex file to
% produce the bibliography for the citations in your paper.
\bibliographystyle{abbrv}
\bibliography{crowdsensing}  % sigproc.bib is the name of the Bibliography in this case
% You must have a proper ".bib" file
%  and remember to run:
% latex bibtex latex latex
% to resolve all references
%
% ACM needs 'a single self-contained file'!
%
%APPENDICES are optional
%\balancecolumns


\end{document}
